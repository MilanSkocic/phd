% !TEX root = ../main_file.tex


% CONSTANTS
\nomenclature[C0102]{h}{Constante de Planck}
\nomenclature[C0103]{c}{Vitesse de la lumière dans le vide}
\nomenclature[C0105]{F}{Constante de Faraday}
\nomenclature[C0106]{R}{Constante universelle des gaz parfaits}
\nomenclature[C0107]{k}{Constante de Boltzmann}
\nomenclature[C0108]{e}{Charge élémentaire}
\nomenclature[C0108]{$\epsilon _0$}{Permittivité du vide}



% ELECTROCHEMISTRY

\nomenclature[E0099]{$\alpha _a$}{Coefficient de transfert anodique}
\nomenclature[E0100]{$\alpha _c$}{Coefficient de transfert cathodique}
\nomenclature[E0101]{$b_a$}{Pente de Tafel pour la branche anodique}
\nomenclature[E0102]{$b_c$}{Pente de Tafel pour la branche cathodique}
\nomenclature[E0103]{z}{Nombre d'électrons échangés}
\nomenclature[E0104]{$j$}{Densité de courant}
\nomenclature[E0105]{$j_0$}{Densité de courant d'échange}
\nomenclature[E0106]{U}{Potentiel électrochimique, mesuré ou appliqué, par rapport à une référence}
\nomenclature[E0107]{$U_{eq}$}{Potentiel électrochimique à l'équilibre mesuré par rapport à une référence}
\nomenclature[E0108]{$U_{fb}$}{Potentiel de bande plate par rapport à une référence}
\nomenclature[E0109]{$\eta$}{Surtension ($U-U_{eq}$)}


% PHOTOELECTROCHESMITRY
\nomenclature[P0101]{$\Phi _{\SC}$}{Potentiel électrique dans le semiconducteur}
\nomenclature[P0102]{$\Phi _{el}$}{Potentiel électrique dans l'électrolyte}
%\nomenclature[P0103]{$\Delta\Phi _{\SC/el}$}{Différence de potentiel entre le semiconducteur et l'électrolyte}
\nomenclature[P0105]{$\epsilon$}{Permittivité diélectrique relative}
\nomenclature[P0106]{$w_{sc}$}{Epaisseur de la charge d'espace}
\nomenclature[P0107]{$w_{\Hm}$}{Epaisseur de la double couche électrochimique}
\nomenclature[P0108]{$L_{\cc}$}{Longueur de diffusion moyenne des porteurs de charge minoritaires}

\nomenclature[P0201]{$\lambda$}{Longueur d'onde de la lumière}
\nomenclature[P0202]{$\nu$}{Fréquence de la lumière}
\nomenclature[P0203]{h$\nu$ ou E}{Energie de la lumière}
\nomenclature[P0204]{$\alpha _{\SC}$}{Coefficient d'absorption du semiconducteur}
\nomenclature[P0205]{$\phi$}{Flux de photon}

\nomenclature[P0301]{$\iph$}{Photocourant tel que mesuré (valeur complexe)}
\nomenclature[P0301]{$\ipht$}{Photocourant rapporté à un flux de photons normalisé (valeur complexe)}
\nomenclature[P0302]{$\theta$}{Phase entre le signal mesuré et le signal de référence correspondant au retard de
l'établissement du photocourant par rapport à l'illumination}
\nomenclature[P0303]{K}{facteur d'amplitude du photocourant correspondant à la pente de la transformée linéaire $(\vert
I_{ph}^{\ast} \cdot \vert h\nu)^{1/2}=f(h\nu - E_g)$ (loi de Gärtner-Butler)}


\nomenclature[P0400]{$\E_g$}{Largeur de bande interdite (gap)}
\nomenclature[P0401]{$\E_F$}{Niveau de Fermi}
\nomenclature[P0402]{$\E_c$}{Niveau d'énergie correspondant au bas de la bande de conduction}
\nomenclature[P0403]{$\E_{cs}$}{Niveau d'énergie correspondant au bas de la bande de conduction en surface}
\nomenclature[P0404]{$\E_v$}{Niveau d'énergie correspondant au haut de la bande de valence}
\nomenclature[P0405]{$\E_{vs}$}{Niveau d'énergie correspondant au haut de la bande de valence en surface}
\nomenclature[P0406]{$\E_d$}{Niveau d'énergie donneur dans le cas d'un dopage de type \emph{n}}
\nomenclature[P0407]{$\E_a$}{Niveau d'énergie accepteur dans le cas d'un dopage de type \emph{p}}
\nomenclature[P0408]{$\E_{fb}$}{Niveau de Fermi en situation de bandes plates}


% ABBREVIATIONS

\nomenclature[A0100]{BWR}{Boiling Water Reactor}
\nomenclature[A0102]{PWR}{Pressurized Water Reactor}
\nomenclature[A0103]{CRUD}{Chalk River Unidentified Deposits}
\nomenclature[A0104]{ESSC}{Enhanced Spacer Shadow Corrosion}
\nomenclature[A0105]{IAEA}{International Atomic Energy Agency}
\nomenclature[A0201]{PEC}{PhotoElectroChimie}
\nomenclature[A0202]{SHE}{Standard Hydrogen Electrode}
\nomenclature[A0203]{SCE}{Saturated Calomel Electrode}
\nomenclature[A0204]{MSE}{Mercury Sulphate Electrode}
\nomenclature[A0205]{OCV}{Open Circuit Voltage}
\nomenclature[A0206]{ZRA}{Zero Resistance Ammeter}
\nomenclature[A0301]{PEEK}{PolyEtherEtherKetone}
\nomenclature[A0302]{PTFE}{PolyTetraFluoroEthylene}
\nomenclature[A0401]{ppm}{Parts Per Million (rapport massique)}
\nomenclature[A0402]{ppb}{Parts Per Billion (rapport massique)}

\renewcommand{\nomname}{Liste des principaux sigles et notations utilisés dans le manuscrit}
\renewcommand{\nomlabel}[1]{#1\hfill}
\printnomenclature


