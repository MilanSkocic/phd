% !TEX root = ../main_file.tex
%%%%%%%%%%%%%%%%%%%%%%%%%%%%%%%%%%%%%%%%%%%%%%%%%%%%%%
%%             Commandes Meta-données               %%
%%   à renseigner par les auteurs pour générer      %%
%%     la couverture modèle Univ. Grenoble          %%
%%%%%%%%%%%%%%%%%%%%%%%%%%%%%%%%%%%%%%%%%%%%%%%%%%%%%%
%%      Fichier encodé au format utf-8        %%

%\Sethpageshift{???mm}   %%optionnel : à décommenter si besoin pour ajout d'espace afin de center la couvérture horizontalement (valeur par défaut est -5.5mm)
%\Setvpageshift{???mm}   %%optionnel : à décommenter si besoin pour ajout d'espace afin de center la couvérture verticalement (valeur par défaut est -15.5mm)


%\Universite{}    %%optionnel : à décommenter et à renseigenr si vous voulez changer le non d'université
%\Grade{}         %%optionnel : à décommenter et à renseigenr si vous voulez changer le grade
\Specialite{Matériaux, Mécanique, Génie Civil, Electrochimie}
\Arrete{6 janvier 2005 - 7 août 2006}
\Auteur{Milan Skocic}
\Directeur{Jean-Pierre Petit}
\CoDirecteur{Yves Wouters}    %%optionnel : à décommenter et à renseigenr si présence d'un Co-directeur de thèse
\Laboratoire{Laboratoire Science et Ingénieurie des matériaux et des Procédés (SIMaP)}
\EcoleDoctorale{Ecole Doctorale I-MEP2}         
\Titre{Etude (photo)-électrochimique en réacteur simulé du phénomène de shadow corrosion des alliages de zirconium}
%\Soustitre{}      %%optionnel : à décommenter et à renseigenr si présence d'un sous-titre de thèse
\Depot{27 mai 2016}       


% Commande pour création de nouvelles catégories dans le jury:

%\UGTNewJuryCategory{...NomDeLaCategorie...}{...Definition...}

% Exemple \UGTNewJuryCategory{UGTFamille}{Membre de la famille} que nous ajoutons dans la commande \Jury ci-dessous sous la forme \UGTFamille{Jean Rousseau}{(...titre_et_affiliation...s'il_y_en_a...)}


\Jury{
\UGTPresident{M., Marian CHATENET, }{Professeur, Université Grenoble Alpes}
%\UGTPresidente{Civilité, Prénom-et-Nom}{titre-et-affiliation}

\UGTRapporteur{M., Sébastien CHEVALIER}{Professeur, Université de Bourgogne}      %% 1er rapporteur
%\UGTRapporteur{M., Hubert PERROT}{Directeur de recherche CNRS, Université P. Curie UMPC-Paris6}      %% second rapporteur
%\UGTRapporteur{M., VIGNAL}{Directeur de recherche CNRS, Université de Bourgogne}      %% second rapporteur
 
\UGTRapporteur{M., Marc TUPIN}{Ingénieur chercheur, CEA Saclay}     %% 1er examinateur
\UGTExaminateur{M., Damien KACZOROWSKI}{Ingénieur expert corrosion, Areva}     %% second examinateur
%\UGTExaminatrice{Civilité, Prénom-et-Nom}{titre-et-affiliation}    %% 3ème examinateur

% Directeur de thèse
\UGTCoDirecteur{M., Yves WOUTERS}{Professeur, Université Grenoble Alpes}     

\UGTDirecteur{M., Jean-Pierre PETIT}{Professeur, Université Grenoble Alpes}{, Invité}
}

\MakeUGthesePDG    %% très important pour générer la couvérture de thèse

