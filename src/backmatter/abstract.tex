% !TEX root = ../main_file.tex
\begin{abstract}


    Des méthodes électrochimiques classiques, et des caractérisations photoélectrochimiques (PEC), utilisées
    \emph{ex-situ} et \emph{in-situ},
    ont permis d’étudier le phénomène de Shadow Corrosion, considéré ici comme une corrosion galvanique entre
    des alliages de zirconium et de nickel, corrosion influencée par l’environnement chimique et l’irradiation de ces
    alliages. Une cellule électrochimique simulant les conditions d’un réacteur à eau bouillante (REB), permettant
    l’illumination UV--Visible des échantillons et le contrôle de la chimie de l’eau, a été conçue, développée et
    validée. Cette cellule a permis de mesurer pour la première fois des spectres en énergie de photocourant d’un
    alliage de zirconium, \emph{in-situ} en milieu REB simulé. Par ailleurs, les résultats expérimentaux obtenus tendent à
    montrer que les impuretés de type cations métalliques jouent un rôle important dans le mécanisme d’activation du
    couplage galvanique, donc potentiellement dans le mécanisme d’activation du phénomène de Shadow Corrosion, alors que
    la présence d’oxygène et/ou de peroxyde d’hydrogène n’induit pas de différences significatives du comportement
    électrochimique des échantillons. Il est montré également que l’illumination UV--Visible des échantillons, qui
    amplifie notablement les courants de couplage, est un paramètre important du phénomène de Shadow Corrosion.

    \vspace{0.5cm}
\noindent Mots clés : Photoélectrochimie \emph{in-situ}, Zircaloy, Alliage de nickel, Conditions REB simulées, Corrosion galvanique


\end{abstract}

\begin{center}
\noindent\rule{\textwidth}{1pt}
\end{center}
\renewcommand{\abstractname}{Abstract}
\begin{abstract}

    Conventional electrochemical methods as well as photoelectrochemical characterizations (PEC), performed
    \emph{ex-situ} et \emph{in-situ}, were used to study the Shadow corrosion phenomenon, considered as a galvanic corrosion between Zr-based and
    Ni-based alloys. The Shadow corrosion is influenced by the chemical environment and the irradiation of these alloys. An
    electrochemical cell , simulating the conditions of a boiling water reactor (BWR), allowing the illumination of the
    samples with UV--Visible as well as monitoring the water chemistry was designed, developed and validated. The cell
    allowed, for the first time, recording of \emph{in-situ} photocurrent energy spectra on a Zr-based alloy in simulated BWR
    environment. Furthermore, the obtained experimental results pointed out that the metallic cation impurities played an
    important role in the activation mechanism of the galvanic coupling, thus potentially in the activation mechanism of the
    Shadow corrosion phenomenon, whereas the presence oxygen and/or hydrogen peroxide did not induce significant differences
    in terms of electrochemical behavior of the samples. It was also shown that the illumination of the sample with
    UV--visible light, which significantly amplified the galvanic current, is an important parameter of the Shadow corrosion
    phenomenon.

\vspace{0.5cm}
\noindent Keywords : \emph{In-situ} Photoelectrochemistry, Zircaloy, Ni-based Alloys, Simulated BWR Conditions, Galvanic Corrosion


\end{abstract}
