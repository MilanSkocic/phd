% !TEX root=../../main_file.tex
% !TeX TS-program = pdflatex
% !TeX checkspelling = fr_toutesvariantes


\chapter*{Conclusion générale}\label{chap:conclusions}
\phantomsection
\addstarredchapter{Conclusion générale}

\begin{refsection}


Au cours de ce travail, nous avons réalisé l’étude expérimentale du phénomène de Shadow Corrosion qui est une forme de
corrosion localisée et aggravée des alliages de zirconium lorsque ces derniers se trouvent à proximité d’alliages plus
nobles tels que les alliages de nickel. Ce phénomène est uniquement observé dans les REB et peut mener à une rupture de
la gaine de combustible. Le mécanisme de couplage galvanique est aujourd’hui assez largement admis dans la communauté
scientifique du nucléaire, de même que l’importance des effets de la chimie de l’eau et de l’irradiation. Nous avons
fait le choix de simuler une partie de l’effet de l’irradiation en réalisant des illuminations avec de la lumière
UV--Visible. Des méthodes électrochimiques classiques ainsi que des caractérisations photoélectrochimiques (PEC) nous ont
permis d’étudier le comportement électrochimique des alliages Zy2 et Inc718 en milieu REB simulé.

Notre travail a consisté, dans un premier temps, à concevoir, développer et valider un dispositif expérimental innovant
permettant de réaliser des mesures photoélectrochimiques \emph{in-situ} en environnement REB simulé. Pour cela, nous avons
sélectionné les sources d’illuminations adéquates, conçu et développé une cellule électrochimique avec un
porte-échantillon original permettant d’illuminer simultanément un échantillon en forme disque et un échantillon en
forme d’anneau. L’illumination se fait à travers un hublot en saphir ayant une géométrie spécifique permettant de
minimiser la distance d’électrolyte que doit traverser la lumière. De plus, nous avons également réalisé un travail
original visant à établir une procédure d’estimation des incertitudes sur les largeurs de bande interdites déduites des
ajustements numériques des spectres en énergie de photocourants. La cellule électrochimique a été couplée avec une
boucle de contrôle de la chimie.

Ce travail de développement nous a conduit à choisir une lampe Hg pour les illuminations UV--Visible continues et une
lampe Xe pour les illuminations UV--Visible modulées. Les deux sources d’illumination nous ont permis de mesurer
l’évolution des potentiels électrochimiques, des courants de couplage et des courbes de polarisation avec et sans
illuminations UV--Visible en contrôlant la chimie de l’électrolyte et enfin de réaliser des premiers spectres en énergie
de photocourants.

%Néanmoins, la régulation de la température en entrée de la cellule électrochimique n’a pas été simple à cause de la
%faible inertie des tubes de circulation. Nous proposons d’ajouter un accumulateur thermique, à l’entrée de la cellule
%électrochimique, ayant une forte inertie thermique afin de minimiser les variations de température et faciliter le
%réglage des paramètres PID. Une approche serait d’utiliser un tube ayant une section plus importante que les tubes de
%circulation avec une paroi plus épaisse. De cette manière, l’électrolyte est ralenti et l’inertie thermique de
%l’accumulateur tamponne les variations de température.


A la suite du travail de conception et de développement, nous avons étudié dans un premier temps, en l’absence
d’illumination UV--Visible des échantillons testés, l’effet de la présence d’impuretés dans l’eau ultra-pure sur la
corrosion d’échantillons de Zy2 en situation de couplage avec des échantillons d’Inc718, aucun de ces échantillons
n’ayant subi une préoxydation préalable. La présence de cations de fer dans l’électrolyte se révèle être néfaste en
favorisant le couplage galvanique alors que ce dernier n’a pas d’effet notable en eau ultra-pure. En effet, les
caractérisations photoélectrochimiques \emph{ex-situ} (post-mortem) indiquent que les cations de fer favorisent la formation
d’une couche d’oxydation plus conductrice dans le cas de l’alliage Inc718. De plus, une synergie des cations de fer avec
les cations de nickel et de zinc semble impacter la couche de zircone formée sur l’alliage de Zy2 en favorisant des
phases semiconductrices dont la largeur de bande interdite est inférieure à 3.5~eV et donc potentiellement plus
conductrice par rapport à la zircone ayant une largeur de bande interdite de 5~eV.

Dans un deuxième temps, nous avons étudié l’effet, sur le comportement d’échantillons de Zy2 et Inc718 ayant subi une
préoxydation préalable, des teneurs en oxygène et peroxyde d’hydrogène dissous dans l’eau ultra-pure, en présence et en
l’absence d’illumination continue UV--Visible. Les couches d'oxyde des deux alliages Zy2 et Inc718 présentent une semiconduction de type
\emph{n} à
280°C avec une augmentation du courant de couplage d’un facteur d’environ 5 sous illumination UV--Visible mais de manière
peu différenciée selon que l’électrolyte contient ou non de l’oxygène et/ou du peroxyde d’hydrogène. L’extrapolation des
courants de couplage à des flux lumineux plus représentatifs des conditions de flux en situation de Shadow
Corrosion en réacteur réel, nous a permis d’obtenir des ordres de grandeur de courant de couplage compatibles avec les
valeurs qu’implique le phénomène de Shadow Corrosion. L’illumination UV--Visible semble seulement modifier les densités
de courant d’échange de l’alliage Zy2 avec un faible impact sur les coefficients de transfert dont l’origine est très
certainement lié à la présence d’une conduction mixte dans les couches d’oxyde impliquant que le comportement de type
diode de l’interface semiconducteur/électrolyte est "atténué". Les amplitudes de photocourant plus faibles à 280°C
qu’à l’ambiante, semblent également confirmer l’hypothèse de l’effet de la conduction mixte sur "l’atténuation" du
comportement de type diode à l’interface semiconducteur/électrolyte.

A la lumière de ces premiers résultats, nous pensons que les impuretés en cations métalliques, notamment les cations de
fer, jouent un rôle de premier ordre dans le mécanisme d’activation du couplage galvanique donc potentiellement dans le
mécanisme d’activation du phénomène de Shadow Corrosion alors que la présence d’oxygène et de peroxyde d’hydrogène
n’induit pas de différence significatives de comportement électrochimique des alliages Zy2 et Inc718. 
La présence d’irradiation UV--Visible joue un
rôle tout aussi important que les impuretés car elle est un facteur amplificateur des 
courants de couplage et donc de la corrosion des alliages de Zy2 en
situation de couplage. 

Nous pouvons proposer l’hypothèse que les différents degrés d’oxydation des cations de fer peuvent favoriser les
échanges électroniques à l’interface oxyde/électrolyte sur les alliages Zy2 et Inc718. Ces échanges peuvent être
facilités par la présence d’illumination UV--Visible apportant des porteurs de chargeurs supplémentaires et autorisant
ainsi des transferts électroniques supplémentaires avec la bande de valence étant donné que les couches d’oxyde formées
sur les échantillons de Zy2 et Inc718 présentent systématiquement une semiconduction de type \emph{n}. Il n’est pas exclu que
les cations de fer peuvent être insérés en substitution dans le réseau de la zircone en extrême surface générant ainsi
des lacunes supplémentaires à cause de la différence de degré d’oxydation du zirconium (+IV) et du fer (+II ou +III).
Ces lacunes supplémentaires peuvent favoriser le transport de l’oxygène et par conséquent favoriser la croissance de la
couche de zircone.

A l’issue de ce travail, nous sommes conscients que les hypothèses formulées demeurent fragiles et nécessitent, pour être
validées, des expériences complémentaires. 
%Il reste en particulier à dupliquer les essais réalisés, sans illumination
%UV--Visible, en micro-autoclave, afin de s’assurer que la légère pollution au fluor détectée lors des premiers essais n’a
%pas eu d’influence significative sur les comportements électrochimiques des alliages Zy2 et Inc718. 
Il nous semble par ailleurs potentiellement intéressant d’orienter les travaux
expérimentaux vers l’étude simultanée de l’effet des impuretés et de l’illumination UV--Visible dans la cellule HTP sur
des échantillons n’ayant subi aucune préoxydation. Enfin, il nous semble nécessaire de noter que l’ensemble des
résultats obtenus dans ce travail repose principalement sur des mesures électrochimiques. Il conviendra bien sûr de les
ré-examiner à la lumière des résultats d’autres techniques de caractérisations \emph{ex-situ}, telles que l’XPS, ou la
spectroscopie Raman.

Ce travail a permis de mettre en avant la faisabilité de la technique de caractérisation PEC \emph{in-situ} offrant la
possibilité de suivre l’évolution des couches d’oxyde en "temps réel" durant l’exposition de différents alliages à
280°C dans la cellule HTP. Il faut mentionner que des caractérisations photoélectrochimiques à 280°C dans un électrolyte
aussi peu conducteur que l’eau ultra-pure n’avait jamais été réalisées jusqu’à présent.


\printbibliography[heading=subbibintoc]
\end{refsection}
